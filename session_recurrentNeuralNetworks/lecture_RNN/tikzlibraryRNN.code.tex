%% Common asset library for recurrent neural networks
%%
%% Modular images and common setup for tikz pictures
%%
\usepackage{graphicx}
\usepackage{xargs}
\usepackage{ifthen}
\usepackage{bm}
\usepackage{etoolbox}
\usepackage{sansmath}


%% Tikz libraries
\usetikzlibrary{automata,arrows,shapes,matrix,trees,calc,fit,backgrounds,math,quotes}


%%%%%%%%%%%%%%%%%%%%%%%%%%%%%%%%%%%%%%%%%%%%%%%%%%
%% Basic units and config
%%%%%%%%%%%%%%%%%%%%%%%%%%%%%%%%%%%%%%%%%%%%%%%%%% 
\def\baseunit{1cm}
%% \def\baseunit{28pt} 1cm ~ 28pt
\def\basenodesep{\baseunit}
\def\baseconsep{\basenodesep/2}
%% node sizes
\def\ionodesize{0.6*\baseunit}


%% Basic config
\tikzset{node distance=\basenodesep}
\tikzset{>=latex}

%% Colors
\tikzset{
  nngreen/.style={thick, rounded corners, fill=green!30},
  nnbox/.style={nngreen, draw=black}
}

%% Font config, labels
\tikzset{
  iolabel/.style={
    font=\bfseries\small\sffamily\sansmath,
  },
  nnlabel/.style={
    font=\bfseries\Huge\sffamily\sansmath
  },
}

%% Simple input/output nodes
\tikzset{
  ionode/.style n args={3}{
    circle, thick,
    minimum size=#1,
    inner sep=0pt,
    outer sep=0pt,
    draw=#2!70,
    fill=#2!30,
    label=[iolabel]center:#3,
  }
}
\tikzset{input/.style={ionode={\ionodesize}{blue}{#1}}}
\tikzset{output/.style={ionode={\ionodesize}{red}{#1}}}
\tikzset{state/.style={ionode={\ionodesize}{purple}{#1}}}
\tikzset{hidden/.style={ionode={\ionodesize}{green}{#1}}}

%% Pointwise operation style
\tikzset{
  pwiselabel/.style={
    white,
    font=\sffamily\bfseries\tiny
  },
  packetlabel/.style={
    white,
    font=\sffamily\bfseries\tiny\sansmath
  },
  pwise/.style={
    rectangle,
    inner sep=2.5pt,
    rounded corners=1pt,
    fill=black,
    label={[pwiselabel]center:#1}
  },
}

%% sigmoid / tanh symbols
\tikzset{
    sigtan/.style n args={3}{
    circle,
    minimum size=#1,
    inner sep=0pt,
    fill=#2!80,
    label={center:\tikz\draw[white, thick]  (-#3, -#3) .. controls (#3, -#3) and (-#3, #3) .. (#3, #3);}
  },
  sigmoid/.style={sigtan={14pt}{red}{4pt}},
  tanh/.style={sigtan={14pt}{blue}{4pt}},
}

%% Legend symbols
\tikzset{
  %% vector connection
  vcon/.style={
    label={center:\tikz\draw[->, thick, >=latex'] (0, 0.2*#1) to[out=300, in=180] (0.2*#1, 0) [out=180, in=60] (0, -0.2#1) to[out=60, in=180] (0.2*#1, 0) -- (0.5#1, 0);}
  },
}


\tikzset{
  znode/.style={inner sep=0pt, outer sep=0pt},
  cnode/.style={znode, node distance=\baseconsep},
  Cnode/.style={znode, node distance=\basenodesep},
}
%%%%%%%%%%%%%%%%%%%%%%%%%%%%%%
%% Perceptron
%%%%%%%%%%%%%%%%%%%%%%%%%%%%%%

%% Mackay style
\tikzset{
  pics/mackayperceptron/.style={
    code = {
      \begin{scope}[node distance=2*\basenodesep]
        \node[ionode={32pt}{black}{{$x_0=1$}}] (x0) {};
        \node[input={$x_1$}, below left of=x0] (x1) {};
        \node[input={$x_2$}, below of=x1] (x2) {};
        \node[below of=x2] (vdots) {$\vdots$};
        \node[input={$x_n$}, below of=vdots] (xn) {};
        
        \node[circle, draw=black, thick, minimum size=32pt, right of=x2] (sum) {$\mathrm{f(a)}$};
        
        \node[output={$y$}, right of=sum, node distance=4*\nodesep cm] (y) {};
        
        \draw[->] (x0) -- (sum) node [midway, right] {$w_0$};
        \draw[->] (x1) -- (sum) node [midway, above] {$w_1$};
        \draw[->] (x2) -- (sum) node [midway, above] {$w_2$};
        \draw[->] (xn) -- (sum) node [midway, right] {$w_n$};
        \draw[->] (sum) -- (y);
      \end{scope}
    }
  }
}

\tikzset{
  pics/nnlayer/.default={3}{input}{},
  pics/nnlayer/.style n args = {3}{
    code = {
      \begin{scope}[yshift=#1*\basenodesep/2]
        \pgfmathparse{#1} \pgfmathresult \let\nnodes\pgfmathresult
        \foreach \n in {1, 2, ..., \nnodes} {
          \ifstrempty{#3}%
          {%
            \node[#2, yshift=-\n*\basenodesep] (n\n) {};
          }{%
            \node[#2={$#3_\n$}, yshift=-\n*\basenodesep] (n\n) {};
          }
        }
      \end{scope}
    }
  }
}
\tikzset{
  pics/nnlayerdots/.default={3}{input}{}{m},
  pics/nnlayerdots/.style n args = {4}{
    code =  {
      \tikzmath{
        int \ninner, \nmax, \nvdot;
        \ninner = #1 - 1;
        \nmax = #1;
        \nall = #1 + 1;
        \offset=\ninner;
      }
      \pic {nnlayer={\ninner}{#2}{#3}};
      \begin{scope}[yshift=\offset*\basenodesep/2]
        \ifstrempty{#3}%
        {%
          \node[#2, yshift=-\nall*\basenodesep] (n\nmax) {};
        }{%
          \node[#2={$#3_{#4}$}, yshift=-\nall*\basenodesep] (n\nmax) {};
        }
        \node[yshift=-\nmax*\basenodesep, text height=2.5ex, text depth=.5ex] (vdots) {$\vdots$};
      \end{scope}
    }
  }
}
%% FIXME: Add possibility to connect one-to-many, many-to-one?
\tikzset{
  pics/connectlayers/.style n args = {4}{
    code = {
      \foreach \i in {1, 2, ..., #1} {
        \foreach \j in {1, ..., #2} {
          \draw[->] (#3n\i) -- (#4n\j);
        }
      }
    }
  },
  pics/connectlayersonetoone/.style n args = {3}{
    code = {
      \foreach \i in {1, 2, ..., #1} {
        \draw[->] (#2n\i) -- (#3n\i);
      }
    }
  }
}
%% Boxed hidden layer
\tikzset{
  pics/boxednnlayer/.default={3}{hidden}{}{m}{green},
  pics/boxednnlayer/.style n args = {5}{
    code = {
      \tikzmath{
        int \nnodes, \offset;
        \offset = #1 + 1;
        %% Offset depends on nodes; if odd:
        \nnodes = #1;
        %% if even
        \nnodes = #1 - 1;
      }
      \begin{scope}[yshift=-\nnodes*\basenodesep / 2]
        \draw[rounded corners, draw=#5!80, fill=#5!50, opacity=0.8] (-\basenodesep/2, \offset * \basenodesep/2) rectangle (\basenodesep/2, -\offset * \basenodesep / 2);
      \end{scope}
      \pic {nnlayerdots={#1}{#2}{#3}{#4}};
    }
  }
}


\tikzset{
  pics/ffn/.style = {
    code = {
      \pic (i_) {nnlayerdots={4}{input}{X}{m}};
      % \node[input={$x_1$}] (x1) {};
      % \node[input={$x_2$}, below of=x1] (x2) {};
      % \node[below of=x2] (xvdots) {$\vdots$};
      % \node[input={$x_m$}, below of=xvdots] (xn) {};
    }
  }
}

%%%%%%%%%%%%%%%%%%%%%%%%%%%%%%%%%%%%%%%%%%%%%%%%%%
%% RNN - smallest unit
%%
%% Input, network (represented as box), output
%%%%%%%%%%%%%%%%%%%%%%%%%%%%%%%%%%%%%%%%%%%%%%%%%%
\def\rnninnerwidth{2*\basenodesep}
\def\rnninnerheight{\basenodesep}
\def\rnnouterwidth{2*\basenodesep}
\def\rnnouterheight{3*\basenodesep}

\tikzset{
  pics/rnn/.style= {
    code = {
      \begin{scope}[xshift=\rnnouterwidth/2, yshift=\rnnouterheight/2]
        \node[nnbox, rectangle, nnlabel, minimum width=\rnninnerwidth, minimum height=\rnninnerheight] (center) at (0, 0) {#1};
        \node[znode, below of=center, node distance=\basenodesep + \baseconsep] (input) {};
        \node[znode, above of=center, node distance=\basenodesep + \baseconsep] (output) {};
        \node[znode, right of=center] (right) {};
        \node[znode, left of=center] (left) {};
      \end{scope}
    }
  },
  pics/rnn/.default={A},
  pics/rnnio/.style n args={3}{
    code = {
      \pic{rnn={#1}};
      \node[input={#2}] (xt) at (input) {};
      \node[output={#3}] (yt) at (output) {};
      \draw[->] (xt) to (center);
      \draw[->] (center) to (yt);
    }
  },
  pics/rnnio/.default={A}{$X_t$}{$Y_t$},
  pics/rnniofolded/.style={
    code = {
      \pic (foldedrnn_) {rnnio};
      \draw[->] (foldedrnn_right) -- ++(\baseconsep, 0) -- ++(0, 1.5*\baseconsep) -- ++(-\sep, 0) -- ++(0, -1.5\baseconsep) --(foldedrnn_left);
    }
  }
}

%%%%%%%%%%%%%%%%%%%%%%%%%%%%%%%%%%%%%%%%%%%%%%%%%%
%% Bare RNN - building block for rnn figures
%% 
%% FIXME: ascii drawing of diagram
%%
%%%%%%%%%%%%%%%%%%%%%%%%%%%%%%%%%%%%%%%%%%%%%%%%%%
\def\basernnwidth{5*\basenodesep}
\def\basernnheight{3*\basenodesep}
\def\basernnfigwidth{7*\basenodesep}
\def\basernnfigheight{5*\basenodesep}
\tikzset{
  pics/basernn/.style = {
    code = {
      \begin{scope}[xshift=\basernnfigwidth/2, yshift=\basernnfigheight/2]
        \draw[thick] (-\basernnfigwidth/2, -\basernnfigheight/2) rectangle (\basernnfigwidth/2, \basernnfigheight/2);
        \draw[thick, rounded corners, fill=green!30] (-\basernnwidth/2, -\basernnheight/2) rectangle (\basernnwidth/2, \basernnheight/2);
        \node (cm) at (0, 0) {};
        \node[cnode, below of=cm] (cb1) {};
        \node[cnode, below of=cb1] (cb2) {};
        \node[cnode, above of=cm] (ca1) {};
        \node[cnode, below of=ca1] (ca2) {};
        \node (x) at (-\basernnfigwidth/2 + 3 * \baseconsep, -\basernnfigheight/2 + \baseconsep) {};
        \node (h) at (\basernnfigwidth/2 - 3 * \baseconsep, \basernnfigheight/2 - \baseconsep) {};
      \end{scope}
    }
  }
}




%%%%%%%%%%%%%%%%%%%%%%%%%%%%%%%%%%%%%%%%%%%%%%%%%%
%% OBSOLETE
%%%%%%%%%%%%%%%%%%%%%%%%%%%%%%%%%%%%%%%%%%%%%%%%%%


%%%%%%%%%%%%%%%%%%%%%%%%%%%%%%
%% Constants, sizes, units, ...
%%%%%%%%%%%%%%%%%%%%%%%%%%%%%%
% \def\baseunit{1cm}
\def\nodesepunit{\baseunit}
\def\rnnwidth{4*\baseunit}
\def\rnnheight{2*\baseunit}
\def\rnnfigwidth{5*\baseunit}
\def\rnnfigheight{4.5*\baseunit}
%% Similarly for lstm and gru


\tikzset{node distance=\nodesepunit}
\tikzset{>=latex}

%% Connection nodes
\tikzset{
  connode2/.style={inner sep=0pt, outer sep=0pt, node distance=\consep cm},
  connode/.style={inner sep=0pt, outer sep=0pt},
  connodew/.style={inner sep=0pt, outer sep=0pt, node distance=1.5*\nodesepunit},
  connodeW/.style={inner sep=0pt, outer sep=0pt, node distance=2*\nodesepunit},
  conline/.style={>=latex, thick}
}

%% Time series observation
\tikzset{
  obs/.style={
    circle, thick,
    minimum size=10pt,
    node distance=50pt
  }
}

%%%%%%%%%%%%%%%%%%%%%%%%%%%%%%%%%%%%%%%%%%%%%%%%%%
%% LSTM command and components
%%%%%%%%%%%%%%%%%%%%%%%%%%%%%%%%%%%%%%%%%%%%%%%%%%
%% Base on https://duckduckgo.com/?t=ffab&q=lstm+illustration&atb=v1-1&iax=images&ia=images&iai=https%3A%2F%2Fwww.herongyang.com%2FNeural-Network%2FLSTM-Long-Short-Term-Memory-github.png
\def\nodesep{0.8}
\def\consep{0.4}


\newcommandx*{\lstm}[5][1=,2=,3=,4=,5=]{%
  \begin{tikzpicture}[>=latex, thick, node distance=\nodesep cm]
    \draw[rounded corners, fill=green!30] (0,0) rectangle (6*\nodesep, 3*\nodesep);
    \node[sigmoid] (s1) at (\nodesep, \nodesep) {};
    \node[sigmoid, right of=s1] (s2) {};
    \node[pwise=X, above of=s1, node distance=1.7*\nodesep cm] (p1) {};
    \node[tanh, right of=s2] (t1) {};
    \node[sigmoid, right of=t1] (s3) {};
    \node[pwise=X, right of=s3] (p2) {};
    \node[tanh, above of=p2] (t2) {};
    \node[pwise=X, above of=t1] (p3) {};
    \node[pwise=+, right of=p1, node distance=2*\nodesep cm] (p4) {};

    %% Cell state
    \node[connode2, node distance=1.5*\nodesep cm, left of=p1, anchor=east] (ctminus1) {#1};
    \node[connode2, node distance=3.5*\nodesep cm, right of=p4, anchor=west] (ct) {#2};
    \draw[shorten >= 0.2*\consep cm] (t2) |- (ct);
    \draw[->] (ctminus1) -- (p1) -- (p4) -- (ct);

    %% Input state
    \node[connode2, below of=s1] (con1) {};
    \node[connode2, left of=con1] (con2) {};

    \ifstrempty{#3}%
    {%
      \node[input={}, node distance=\nodesep cm, below of=con2, draw=none, fill=none] (xt) {};      
    }{%
      \node[input={#3}, node distance=\nodesep cm, below of=con2] (xt) {};
    }

    \draw (xt) -- ($ (xt) !.9! (con2) $) to [out=90, in=180] ($ (xt|-con1.east) !.1! (con1.east) $) -- (con1) -| (s1);

    %% Hidden state
    \node[connode2, node distance=\consep * 2 cm, left of=con2, anchor=east] (htminus1) {#4};
    \node[connode2, node distance=\nodesep * 6 cm, right of=con2, anchor=west] (ht) {#5};
    \draw (htminus1) -| (s1);
    \draw (htminus1) -| (t1);
    \draw (htminus1) -| (s2);
    \draw (htminus1) -| (s3);

    %% Remaining edges
    \draw[->] (s1) -- (p1);
    \draw[->] (s2) |- (p3);
    \draw (t1) -- (p3);
    \draw[->] (s3) -- (p2);
    \draw (p2) -- (t2);
    \draw[->] (p2) |- (ht);
    \draw[->] (p3) -- (p4);
    
  \end{tikzpicture}  
}

\newcommand*{\lstmforgetgate}{
   \begin{tikzpicture}
     \node[anchor=west] {\lstm};
     \node (rect) at (1.7*\nodesep, 0.65*\nodesep) [draw=red, thick, dotted, minimum width=0.7*\nodesep cm, minimum height=2.25*\nodesep cm]  {};
     \node[above of=rect, red, font=\sffamily\tiny\bfseries, anchor=south] {forget gate};
   \end{tikzpicture}
}

\newcommand*{\lstminputgate}{
   \begin{tikzpicture}
     \node[anchor=west] {\lstm};
     \node (rect) at (2.7*\nodesep, 0.25*\nodesep) [draw=red, thick, dotted, minimum width=0.7*\nodesep cm, minimum height=1.5*\nodesep cm]  {};
     \node[above of=rect, red, font=\sffamily\tiny\bfseries, anchor=north] {input gate};
   \end{tikzpicture}
}

\newcommand*{\lstmcellstate}{
   \begin{tikzpicture}
     \node[anchor=west] {\lstm};
     \node (rect) at (0.2*\nodesep, 1.6*\nodesep) [draw=red, thick, dotted, minimum width=7*\nodesep cm, minimum height=0.4*\nodesep cm, anchor=west]  {};
     \node[below of=rect, red, font=\sffamily\tiny\bfseries, anchor=north west, node distance=0.2*\nodesep cm] {cell state};
   \end{tikzpicture}
}

\newcommand*{\lstmoutputgate}{
   \begin{tikzpicture}
     \node[anchor=west] {\lstm};
     \node (rect) at (4.25*\nodesep, -0.25 * \nodesep) [draw=red, thick, dotted, minimum width=1.25*\nodesep cm, minimum height=\nodesep cm, anchor=west]  {};
     \node[above of=rect, red, font=\sffamily\tiny\bfseries, anchor=south, node distance=0.5*\nodesep cm] {output gate};
   \end{tikzpicture}
}

%%%%%%%%%%%%%%%%%%%%%%%%%%%%%%%%%%%%%%%%%%%%%%%%%% 
%% GRU
%%%%%%%%%%%%%%%%%%%%%%%%%%%%%%%%%%%%%%%%%%%%%%%%%%
\newcommandx*{\gru}[3][1=,2=,3=]{
  \begin{tikzpicture}[>=latex, thick, node distance=\nodesep cm]
    \draw[rounded corners, fill=green!30] (0,0) rectangle (6*\nodesep, 3*\nodesep);
    \node[sigmoid] (s1) at (2*\nodesep, 1.25*\nodesep) {};
    \node[sigmoid, right of=s1] (s2) {};
    \node[connode2, above of=s1, node distance=1.5*\consep cm] (cs1p1) {};
    \node[pwise=X, left of=cs1p1] (p1) {};
    \node[pwise={1-}, above of=s2] (p2) {};
    \node[pwise={X}, above of=p2, node distance=\consep cm] (p3) {};
    \node[tanh, right of=s2, node distance=1.5*\nodesep cm] (t1) {};
    \node[pwise=X, above of=t1, node distance=1.5*\consep cm] (p4) {};
    \node[pwise=+, right of=p3, node distance=1.5*\nodesep cm] (p5) {};

    %% Invisible nodes
    \node[connode2, above of=s2, node distance=1.5*\consep cm] (p2b) {};
    % Row 2
    \node[connode2, below of=s1] (s32) {};
    \node[connode2, left of=s32, node distance=\nodesep cm] (s22) {};
    \node[connode2, left of=s22] (s12) {};

    % Row 1
    \node[connode2, below of=s12] (s11) {};
    \node[connode2, right of=s11] (s21) {};
    \node[connode2, right of=s21] (s31) {};
    \node[connode2, right of=s21, node distance=3*\nodesep cm] (s51) {};
    
    %% Input
    \ifstrempty{#3}%
    {%
      \node[input={}, node distance=.75*\nodesep cm, below of=s11, draw=none, fill=none] (xt) {};      
    }{%
      \node[input={#3}, node distance=.75*\nodesep cm, below of=s11] (xt) {};
    }
    %%\node[input={$\boldsymbol{x_t}$}, node distance=.75*\nodesep cm, below of=s11] (xt) {};
    
    %% Cell state
    \node[connode2, node distance=3.5*\nodesep cm, left of=p3, anchor=east] (ctminus1) {#1};
    \node[connode2, node distance=2*\nodesep cm, right of=p5, anchor=west] (ct) {#2};
    \draw[->] (ctminus1) -- (p3) -- (p5) -- (ct);

    %% Edges
    
    % Tanh to +
    \draw[->] (p4) -- (p5);
    \draw (t1) -- (p4);

    % xt to t1 and others
    \draw (xt) |- (s11-|t1) -| (t1);
    \draw (xt) -- ($ (xt) !.9! (s12) $) to [out=90, in=180] ($ (xt|-s22.east) !.1! (s22.east) $) -| (s1);
    \draw (ctminus1) -| ($ (xt) !1.1! (s12) $) to [out=270, in=180] ($ (xt|-s22.east) !.1! (s22.east) $) -| (s1);
    \draw (s32) -| (s2);
    \draw[->] (s1) |- (p1);
    \draw (ctminus1) -| (p1);
    \draw[shorten >= 0.05*\consep cm] (p1) -- (s22);
    \draw[shorten <= 0.05*\consep cm] (s22) -- ($ (s22) !.9! (s21) $) to[out=270, in=180] ($ (s21) !.1! (s31) $);
    \draw (s2) -- (p2);
    \draw[->] (p2) -- (p3);
    \draw[->] (p2b) -- (p4);

  \end{tikzpicture}  
}
%%%%%%%%%%%%%%%%%%%%%%%%%%%%%%%%%%%%%%%%%%%%%%%%%%
%% Vanilla RNNs
%%%%%%%%%%%%%%%%%%%%%%%%%%%%%%%%%%%%%%%%%%%%%%%%%%
\tikzset{
  xtlabel/.initial = $X_t$,
  htlabel/.initial  = $H_t$,
  pics/vanillarnn/.style 2 args = {
    code = {
      \tikzset{xtlabel/.get=\xtlabel}
      \tikzset{htlabel/.get=\htlabel}
      \begin{scope}[xshift=\rnnfigwidth/2, yshift=\rnnfigheight/2]
        \draw[thick] (-\rnnfigwidth/2, -\rnnfigheight/2) rectangle (\rnnfigwidth/2, \rnnfigheight/2);
        \draw[thick, rounded corners, fill=green!30] (-\rnnwidth/2, -\rnnheight/2) rectangle (\rnnwidth/2, \rnnheight/2);
        \node[tanh] (t1) at (0, 0) {};
        \node[input={\xtlabel}] (xt) at (-\rnnwidth/2 + \baseunit/2, -\rnnheight/2 - 3*\baseunit/4) {};    
        \node[output={\htlabel}] (ht) at (\rnnwidth/2 - \baseunit/2, \rnnheight/2 + 3*\baseunit/4) {};

        %% Connection nodes
        \node[connodew, above of=xt] (xta) {};
        \node[connode, below of=t1] (t1b) {};
        \node[connode, above of=t1] (t1a) {};
        \node[connodew, below of=ht] (htb) {};
        \node[connode] (x) at ($ (t1a) !.5! (htb) $) {};
        \node[connodew, right of=x] (htr) {};        

        \draw[->, conline] (xt) |- ($ (xta) !.5! (t1b) $) -| (t1);
        \draw[->, conline] (t1) |- ($ (t1a) !.5! (htb) $) -| (ht);
        \draw[->, conline] (x) -- (htr);
      \end{scope}
    }
  },
  vanillarnnoverlay/.pic={%%
    \pic{vanillarnn};
    \begin{scope}[xshift=\rnnfigwidth/2, yshift=\rnnfigheight/2]
      \draw[ultra thick, rounded corners, fill=green!30, opacity=0.8] (-\rnnwidth/2, -\rnnheight/2) rectangle (\rnnwidth/2, \rnnheight/2);
      \node[font=\sffamily\Huge] (A) at (0, 0) {A};
    \end{scope}
  }
}


